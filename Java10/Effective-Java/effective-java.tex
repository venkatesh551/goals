
%----------------------------------------------------------------------------------------
%	PACKAGES AND THEMES
%----------------------------------------------------------------------------------------

\documentclass{beamer}

\usetheme{Warsaw}

\usepackage{listings}
\usepackage[utf8]{inputenc}
\usepackage{url}
\setbeamertemplate{itemize item}[triangle]
%----------------------------------------------------------------------------------------
%	TITLE PAGE
%----------------------------------------------------------------------------------------

\title{Lambda Expressions} % The short title appears at the bottom of every slide, the full title is only on the title page

\author{Venkatesh} % Your name
\institute{WDC}
\date{\today} % Date, can be changed to a custom date

\begin{document}

\lstset{
    basicstyle=\ttfamily\footnotesize,
    breaklines=true
    breakatwhitespace=true,
    language=C++,
    columns=fullflexible,
    keepspaces=true,
    breaklines=true,
    tabsize=3, 
    showstringspaces=false,
    extendedchars=true
    inputencoding=utf8
}

\begin{frame}
\titlepage % Print the title page as the first slide
\end{frame}

\begin{frame}{Outline}

  \begin{enumerate}

   \item Introduction
   \item Anonymous Inner class
   \item Functional Interfaces
   \item Lambda Expressions
   \item Anonymous classes vs Lambdas
   
  \end{enumerate}
\end{frame}


%----------------------------------------------------------------------------------------
%	PRESENTATION SLIDES
%----------------------------------------------------------------------------------------

%----------------------------------------------------------------------------------------

\begin{frame}[fragile]
\frametitle{Introduction}

Anonymous Classes
\begin{itemize}
  \item <1-> shows you how to implement a base class or \texttt{interface} without giving it a name
  \item <2-> enable you to declare and instantiate a \texttt{class} at the same time
  \item <3-> If an interface contains only one method, then the syntax of anonymous classes may seem unwieldy and unclear
\end{itemize}

\end{frame}


\begin{frame}[fragile]
\frametitle{Anonymous class}

\begin{itemize}
\item <1-> It is always an inner class, it is never static
\item <2-> can implement only one \texttt{interface} at a time
\item <3-> can extend a \texttt{class} (may be abstract or concrete) only one at a time
\item <4-> cannot have an explicitly declared constructor
\end{itemize}

\end{frame}

\begin{frame}[fragile]
\frametitle{Anonymous Inner class}

\begin{block}{Syntax}
\texttt{new} \textit{[TypeArguments]} \\ 
\hspace{6mm} \textit{ClassOrInterfaceTypeToInstantiate} ( \textit{[ArgumentList]} ) \\
\hspace{6mm} \{\textit{ClassBody}\}
\end{block}
\pause
\begin{block} {capture variables}
can access local variables in its enclosing scope that are \texttt{final} or "effectively final"
\end{block}

\end{frame}


\begin{frame}[fragile]
\frametitle{Anonymous class that \texttt{extends} a class}

\begin{example}
    \begin{lstlisting}
// Here we are using Anonymous Inner class 
// that extends a class i.e. Here a Thread class 

Thread t = new Thread() { 
      public void run() { 
          System.out.println("Child Thread"); 
      } 
  }; 
t.start(); 

\end{lstlisting}
\end{example}

\end{frame}


\begin{frame}[fragile]
\frametitle{Anonymous class that \texttt{implements} a interface}

\begin{example}
    \begin{lstlisting}
  // Here we are using Anonymous Inner class 
  // that implements a interface i.e. Here Runnable interface

  Runnable r = new Runnable() { 
      public void run() { 
          System.out.println("Child Thread"); 
      }
  };
  Thread t = new Thread(r); 
  t.start();  

  \end{lstlisting}
\end{example}
\end{frame}


\begin{frame}[fragile]
\frametitle{Functional Interface}

\begin{block} {Definition}
    
An interface that contains \textit{one and only one} abstract method
(which is not public method of \texttt{Object})
\end{block}
\pause
\begin{example}
    \begin{lstlisting}
interface Runnable {
  void run();
}
\end{lstlisting}
\end{example}

\end{frame}


\begin{frame}[fragile]
\frametitle{Functional Interface Examples}

\begin{example}
    \begin{lstlisting}
/* Non Functional : equals is a member of Object */    
interface NonFunc {
  boolean equals(Object obj);
}
\end{lstlisting}
\pause
\begin{lstlisting}
/* Functional Interface : an abstract method which is not a member of Object */
interface Func extends NonFunc {
  int compare(String o1, String o2);
}
\end{lstlisting}
\pause
\begin{lstlisting}
/* java.util.Comparator<T> is functional because it has one abstract non-Object method */
interface Comparator<T> {
  boolean equals(Object obj);
  int compare(T o1, T o2);
}

\end{lstlisting}
\end{example}
\end{frame}


\begin{frame}[fragile]
    \frametitle{Creating Instances of Functional Interface}
    \begin{itemize}
    \item<1-> Declaring and instantiating a class that \texttt{implements} the interface
    \item<2-> Lambda expressions
    \item<3-> Method reference expressions
    \end{itemize}
\end{frame}


\begin{frame}[fragile]
    \frametitle{Lambda Expressions}

\begin{itemize}
  \item A lambda expression is like a method
  \item It forms the implementation of the abstract method defined by functional interface
\end{itemize}

\begin{block} {LambdaExpression Syntax:}
    \texttt{LambdaParameters -> LambdaBody}
\end{block}

\end{frame}


\begin{frame}[fragile]
    \frametitle{Lambda Expressions ...}

\begin{itemize}
   \item <1-> Lambda expressions are always poly expressions (not standalone expressions)
   \item <2-> Value of a lambda-expr is an instance of a class 
              that \texttt{implements} the functional interface type
\end{itemize}

\begin{block} {Lambda expression can occurs only in}
  \begin{itemize}
  \item assignment context
  \item invocation context or
  \item casting context
  \end{itemize}
\end{block}

\end{frame}


\begin{frame}[fragile]
    \frametitle{Lambda Parameters}
\begin{block}{formal parameters of a lambda expression}
\begin{itemize} 
\item either \textit{declared types} or \textit{inferred types}
\item styles cannot be mixed
\item Only parameters with declared types can have modifiers
\end{itemize}
\end{block}
\end{frame}

\begin{frame}[fragile]
    \frametitle{Lambda Body}
\begin{itemize} 
\item <1-> A lambda body is either a \textit{single expression} or \textit{a block}
\item <2-> \texttt{this} and \texttt{super} keywords in a lambda body are same as in the surrounding context
\item <3-> It can't use a local variable/formal parameter of its enclosing scope unless 
that variable is \texttt{final} or "effectively final"
\item <4-> It cannot modify the local variable/formal parameter of the enclosing scope
\end{itemize}
\end{frame}


\begin{frame}[fragile]
    \frametitle{Lambda Expressions ...}
\begin{itemize}
   \item <1-> Type of the \textit{abstract method} and the type of the lambda expression must be compatible
   \item <2-> Type and number of the lambda parameters must be comptaible with method's parameters
   \item <3-> Any \textit{checked exceptions} that can be thrown in the body of a lambda must be declared
\end{itemize}
\end{frame}


\begin{frame}[fragile]
    \frametitle{Lambda Expressions Syntax Examples }

\begin{lstlisting}
() -> 42  // No parameters, expression body
() -> {return 42;}  // No parameters, block body with return
\end{lstlisting}
\pause

\begin{lstlisting}
(int x, int y) -> x + y  // Multiple declared-type parameters
(x, y) -> x + y  // Multiple inferred-type parameters
\end{lstlisting}
\pause

\begin{lstlisting}
(x, int y) -> { return x + y; } /* Illegal: can't mix inferred and declared types */
(x, final y) -> x+y // Illegal: no modifiers with inferred types
\end{lstlisting}

\end{frame}



\begin{frame}[fragile]
\frametitle{Example}

\begin{lstlisting}
interface MyNumber {
  double getValue();
}

// Create a reference to a MyNumber instance.
MyNumber myNum;

// Use a lambda in an assignment context.
myNum = () -> 123.45;

/* Call getValue(), which is implemented by the previously assigned lambda expression. */
System.out.println(myNum.getValue());

\end{lstlisting}
\end{frame}


\begin{frame}[fragile]
\frametitle{More Examples}

\begin{itemize}
  \item <1-> BlockLambdaDemo.java
  \item <2-> LambdaExceptionDemo.java
  \item <3-> VarCapture.java
  \item <4-> VarCapture-2.java

\end{itemize}

\end{frame}

\begin{frame}[fragile]
\frametitle{Standard Functional Interfaces }
\begin{itemize}
\item \texttt{java.util.function} provides several predefined functional interfaces
\end{itemize}

\begin{table}
\begin{tabular}{|l    l|} \hline

\textbf{Interface} & \textbf{Function signature} \\ \hline
\texttt{UnaryOperator<T>}  & \texttt{T	apply(T	t)} \\
\texttt{BinaryOperator<T>} & \texttt{T	apply(T	t1, T	t2)} \\
\texttt{Predicate<T>}      & \texttt{boolean test(T	t)} \\  
\texttt{Function<T,R>}     & \texttt{R	apply(T	t)} \\ 
\texttt{Supplier<T>}       & \texttt{T	get()} \\ 
\texttt{Consumer<T>}       & \texttt{void	accept(T t)} \\ \hline

\end{tabular}
\end{table}
\end{frame}

\begin{frame}[fragile]
\frametitle{Example}

\begin{itemize}
  \item <1-> UseFunctionInterfaceDemo.java
\end{itemize}


\end{frame}



\begin{frame}[fragile]
\frametitle{Type of a Lambda Expression}

A lambda of the form \texttt{() -> stmt-expr} is interpreted as either
\begin{lstlisting} 
() -> { return stmt-expr; }
\end{lstlisting}
or
\begin{lstlisting} 
() -> { stmt-expr; }  
\end{lstlisting}
depending on the target type


\begin {example}
\begin{lstlisting}
// Predicate has a boolean result
java.util.function.Predicate<String> p = s -> list.add(s);
// Consumer has a void result
java.util.function.Consumer<String> c = s -> list.add(s);
\end{lstlisting}
\end{example}

\end{frame}


\begin{frame}[fragile]
\frametitle{Anonymous classes vs Lambdas}

\begin{itemize}
\item <1-> Lambdas are limited to functional interface
\item <2-> Using an anonymous class, we can create an instance of
  \begin{itemize}
  \item an abstract class
  \item interface with multiple abstract methods
  \end{itemize}
\item <3-> In anonymous class, \texttt{this} refers to class instance itself
\end{itemize}

\end{frame}


\begin{frame}[fragile]
\frametitle{Effective Java }

\begin{itemize}
\item <1-> Prefer lambdas to anonymous	classes
\item <2-> Favor the use of standard functional interfaces
\end{itemize}

\end{frame}










%------------------------------------------------
\begin{frame}
\frametitle{References}
\begin{thebibliography}{2} % Beamer does not support BibTeX so references must be inserted manually as below
\setbeamertemplate{bibliography item}[book]

\bibitem{Version 5.2.0} Java The Complete Reference 9th Edition, Herbert Schildt
\bibitem{SE10} The Java Language Specification Java SE 10 Edition
\bibitem{SE10} Effective Java, 3rd Edition, Joshua Bloch
\end{thebibliography}
\end{frame}

%------------------------------------------------

\begin{frame}
\Huge{\centerline{Thank You}}
\end{frame}


%----------------------------------------------------------------------------------------

\end{document} 

