
%----------------------------------------------------------------------------------------
%	PACKAGES AND THEMES
%----------------------------------------------------------------------------------------

\documentclass{beamer}

\usetheme{Warsaw}

\usepackage{listings}
\usepackage[utf8]{inputenc}
\usepackage{url}
\setbeamertemplate{itemize item}[triangle]
%----------------------------------------------------------------------------------------
%	TITLE PAGE
%----------------------------------------------------------------------------------------

\title{Operator Overloading} % The short title appears at the bottom of every slide, the full title is only on the title page

\author{Venkatesh} % Your name
\institute{WDC}
\date{\today} % Date, can be changed to a custom date

\begin{document}

\lstset{
    basicstyle=\ttfamily\footnotesize,
    breaklines=true
    breakatwhitespace=true,
    language=C++,
    columns=fullflexible,
    keepspaces=true,
    breaklines=true,
    tabsize=3, 
    showstringspaces=false,
    extendedchars=true
    inputencoding=utf8
}

\begin{frame}
\titlepage % Print the title page as the first slide
\end{frame}

\begin{frame}{Outline}
  % You might wish to add the option [pausesections]
  \begin{enumerate}
   \item Overloading
   \item Operator Overloading
        \begin{itemize}
        \item Introduction
        \item Special Operators
        \item Type Conversions
        \item Userdefined Literals
        \end{itemize}
  \end{enumerate}
\end{frame}


%----------------------------------------------------------------------------------------
%	PRESENTATION SLIDES
%----------------------------------------------------------------------------------------

%----------------------------------------------------------------------------------------

\begin{frame}[fragile]
\frametitle{Overloading}

\begin{itemize}
\item Two or more declarations for the \texttt{same name} in the \texttt{same scope}
\item Only Function and Function Templates can be overloaded
\item Variable and Type declarations cannot be overloaded
\end{itemize}

\end{frame}

\begin{frame}[fragile]
\frametitle{Non Overloadable Declarations}

\begin{itemize}
\item Functions differ only in return Type
\item static and non-static member functions with same name and parameter-type list
\begin{example}
\begin{lstlisting}
class X {
static void f();
void f();                    // ill-formed
void f() const;              // ill-formed
void f() const volatile;     // ill-formed
void g();
void g() const;              // OK: no static g
void g() const volatile;     // OK: no static g
};

\end{lstlisting}
\end{example}
\end{itemize}
\end{frame}


\begin{frame}[fragile]
\frametitle{Non Overloadable Declarations...}

\begin{itemize}
\item Member functions with same parameter-type list, some are with a ref-qualifier
\end{itemize}
\begin{example}
\begin{lstlisting}
class Y {
void h() &;
void h() const &; // OK
void h() && ;    // OK, all declarations have a ref-qualifier  
void i()  &;
void i() const;  // ill-formed, prior declaration of i 
                 // has a ref-qualifier
};
\end{lstlisting}
\end{example}

\end{frame}



\begin{frame}[fragile]
\frametitle{Non Overloadable Declarations...}

\begin{itemize}
\item Declarations that differ only in the type specifiers are equivalent
\item Note : For any type T "pointer to T", "pointer to const T" are distinct
\end{itemize}
\begin{example}
\begin{lstlisting}
typedef const int cInt;

int f (int);
int f (const int);          //  redeclaration of f(int)

int f (int) { /* ... */ }   //  definition of f(int)
int f (cInt) { /* ... */ }  // error: redefinition of f(int)
\end{lstlisting}
\end{example}
\end{frame}


\begin{frame}[fragile]
\frametitle{Non Overloadable Declarations...}

\begin{itemize}
\item Functions with same name in base and derived classes
\end{itemize}

\begin{example}
\begin{lstlisting}
struct B {
    int f(int);
};
struct D : B {
    int f(const char*); // hides B::f(int)
};
void h(D* pd) {
    pd->f(1);      // error : D::f(const char*) hides B::f(int)
    pd->B::f(1);   // OK
    pd->f("Ben");  // OK, calls D::f
}
\end{lstlisting}
\end{example}
\end{frame}


\begin{frame}[fragile]
\frametitle{Access Rules}
\begin{itemize}
\item overloaded member functions can have different access rules
\end{itemize}

\begin{example}
\begin{lstlisting}
class buffer {
private:
    char* p;
    int size;
protected:
    buffer(int s, char* store) { size = s; p = store; }
public:
    buffer(int s) { p = new char[size = s]; }
};
\end{lstlisting}
\end{example}
\end{frame}


\begin{frame}[fragile]
\frametitle{Overloaded operators}

\begin{block}{syntax}
\begin{lstlisting}
return-type operator symbol(params)
\end{lstlisting}
\end{block}

\begin{block}{symbol: one of}
\begin{lstlisting}
new delete new[] delete[]
+   -   *      /     %     ^    &     |    ~
!   <     >    +=    -=   *=    /=   %=  
^=  &=  |=   <<    >>    >>=    <<=    ==   !=
<=  >=   &&   ||    ++   --    ,   ->*    
=  ->  ()  []
\end{lstlisting}
\end{block}

\end{frame}


\begin{frame}[fragile]
\frametitle{Constraints}

\begin{itemize}
\item Both unary and binary forms of \texttt{+  -  *  \&} can be overloaded
\item We cannot introduce new tokens as operators
\item Precedence, grouping, number of operands cannot be changed
\item Semantics/Identity can be changed
\end{itemize}
\begin{block}{operators cannot be overloaded}
\begin{lstlisting}
  .  .*  ::  ?:  
  #   ##    preprocessing symbols
  sizeof  alignof  typeid
\end{lstlisting}
\end{block}
\end{frame}


\begin{frame}[fragile]
\frametitle{Operator Overloading Rules}
\begin{itemize}
\item Either \texttt{non-static member} function or non-member function
\item Atleast one paramter type is a class/enum
\item Cannot have default arguments
\item \texttt{ = \& (unary) , (comma)} predefined for each type, can be changed
\item \texttt{ = [] () -> } must be non-static member functions
\end{itemize}

\end{frame}


\begin{frame}[fragile]
\frametitle{Operator Overloading}

\begin{example}
\begin{lstlisting}
class X {
public:
    X(int);
    void operator+(int);
};
void operator+(X,X);
void operator+(X,double);
void f(X a) {
    a+1;    // same as  a.operator+(1)
    1+a;    // ::operator+(X(1),a)
    a+1.0;  // ::operator+(a,1.0)
    std::string s = "a" + "b"   // error : both are const char *
}
\end{lstlisting}
\end{example}

\end{frame}


\begin{frame}[fragile]
\frametitle{Operator overload Lookup}

\begin{block}{overload resolution}
No preference is given to members over nonmembers
\end{block}

\end{frame}

\begin{frame}[fragile]
\frametitle{How to Resolve Operators in Namespaces}
\begin{example}
\begin{lstlisting}
#include <iostream>

int main() {
    std::string s = "hello wolrd";
    std::cout << s; // << is defined in namespace std
    return 0;
}

// std::cout.operator<<(s) or operator(std::cout, s)

\end{lstlisting}
\end{example}


\end{frame}

\begin{frame}[fragile]
\frametitle{Operators in Namespaces}
Consider a binary operator @, x@y is resolved like this: \\
x is of type X and y is of type Y.  \\
Look for declarations of operator@
\begin{itemize}
\item if X is a class, check for members of X or base of X ; and
\item context surrounding x@y ; and
\item if X is defined in namespace N, then in N ; and
\item if Y is defined in namespace M, then in M 
\end{itemize}

\end{frame}



\begin{frame}[fragile]
\frametitle{Assignment operator}
\begin{itemize}
\item \texttt{operator=} is a non-static member function with exactly one parameter
\item implicitly declared for a class if not declared by the user
\item Any assignment operator can be virtual
\end{itemize}

\end{frame}

\begin{frame}[fragile]
\frametitle{Function call}
\begin{itemize}
\item \texttt{operator()} is a non-static member function with an arbitrary number of parameters
\end{itemize}
\end{frame}

\begin{frame}[fragile]
\frametitle{Function call operator}
\begin{example}
\begin{lstlisting}
class Action {
public:
    Action();
    int operator()(int);
    pair<int,int> operator()(int,int);
    double operator()(double);
};
void f(Action act)
{
    int x = act(2);
    auto y = act(3,4);
    double z = act(2.3);
};
\end{lstlisting}
\end{example}
\end{frame}


\begin{frame}[fragile]
\frametitle{Lambda Functions}
\begin{itemize}

\item shorthand for defining and using a function object
\item By default, \texttt{operator()} is const, it doesn't modify the captured variables

\end{itemize}

\end{frame}


\begin{frame}[fragile]
\frametitle{Subscripting : \texttt{operator[]} }
\begin{itemize}

\item non-static member function with exactly one parameter
\end{itemize}
\begin{example}
\begin{lstlisting}
struct Assoc {
    vector<pair<string,int>> vec; // vector of {name,value} pairs
    const int& operator[] (const string&) const;
    int& operator[](const string&);
};
Assoc values;
values[string("key")];
\end{lstlisting}
\end{example}
\end{frame}


\begin{frame}[fragile]
\frametitle{Dereferencing: \texttt{operator->} }
\begin{itemize}
\item non-static member function taking no parameters
\end{itemize}
\begin{example}
\begin{lstlisting}
class Ptr {
    X* operator-> ();
};
void g(Ptr p) {
    p->m = 7;               // (p.operator->())->m = 7
    X* q1 = p->;            // syntax error
    X* q2 = p.operator->(); // OK
}
\end{lstlisting}
\end{example}
\end{frame}



\begin{frame}[fragile]
\frametitle{Increment and Decrement }

\begin{example}
\begin{lstlisting}
struct X {
    X& operator++();     // prefix ++
    X operator++(int);   // postfix ++
};
struct Y { };
Y& operator++(Y&);       // prefix ++b 
Y operator++(Y&, int);   // postfix b++

void f(X a, Y b) {
    ++a; // a.operator++();
    a++; // a.operator++(0);
    ++b; // operator++(b);
    b++; // operator++(b, 0);
}

\end{lstlisting}
\end{example}
\end{frame}

\begin{frame}[fragile]
\frametitle{Conversion Operators}
\begin{itemize}
\item conversion from a user-defined type to a built-in type
\item X::operator T()  where T is a type name, defines a conversion from X to T
\end{itemize}
\begin{example}
\begin{lstlisting}
class unique_ptr {
public:
    explicit operator bool() const noexcept;
};
void use(unique_ptr<Record> p, unique_ptr<int> q) {
    if (p) {
        // use it
    }
    bool b = p;     // error ; suspicious use
    int x = p + q;  // error ; we definitly don't want this
}
\end{lstlisting}
\end{example}

\end{frame}


\begin{frame}[fragile]
\frametitle{UserDefined Literals}
\begin{block}{syntax}
operator "" identifier(parameter-declaration-clause)
\end{block}
\begin{lstlisting}
/* identifier is literal suffix identifier */
parameter-declaration-clause is one of :
    const char*
    unsigned long long int
    long double
    char
    const char*, std::size_t
\end{lstlisting}
\begin{example}
\begin{lstlisting}
long double operator "" _km(long double);
\end{lstlisting}
\end{example}
\end{frame}




%------------------------------------------------
\begin{frame}
\frametitle{References}
\begin{thebibliography}{2} % Beamer does not support BibTeX so references must be inserted manually as below
\setbeamertemplate{bibliography item}[book]
\bibitem{Stroustrup} The C++ Programming Language [4th Edition] - Bjarne Stroustrup
\end{thebibliography}
\end{frame}

%------------------------------------------------

\begin{frame}
\Huge{\centerline{Thank You}}
\end{frame}

%----------------------------------------------------------------------------------------




%----------------------------------------------------------------------------------------

\end{document} 

