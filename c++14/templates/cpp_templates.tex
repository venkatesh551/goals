
%----------------------------------------------------------------------------------------
%	PACKAGES AND THEMES
%----------------------------------------------------------------------------------------

\documentclass{beamer}

\usetheme{Warsaw}

\usepackage{listings}
\usepackage[utf8]{inputenc}
\usepackage{url}
\setbeamertemplate{itemize item}[triangle]
%----------------------------------------------------------------------------------------
%	TITLE PAGE
%----------------------------------------------------------------------------------------

\title{C++ Templates} % The short title appears at the bottom of every slide, the full title is only on the title page

\author{Venkatesh} % Your name
\institute{WDC}
\date{\today} % Date, can be changed to a custom date

\begin{document}

\lstset{breakatwhitespace=true,
language=C++,
columns=fullflexible,
keepspaces=true,
breaklines=true,
tabsize=3, 
showstringspaces=false,
extendedchars=true
inputencoding=utf8}


\begin{frame}
\titlepage % Print the title page as the first slide
\end{frame}

\begin{frame}{Outline}
  \tableofcontents
  % You might wish to add the option [pausesections]
\end{frame}


%----------------------------------------------------------------------------------------
%	PRESENTATION SLIDES
%----------------------------------------------------------------------------------------

\section{Introduction}
\begin{frame}{Introduction}

Templates are introduced for the purpose of design,
implementation, and use of the standard library

\begin{block}{C++ STL}
string, ostream, regex, complex, list, map, thread
\end{block}
\end{frame}

\begin{frame}{Templates}

\begin{itemize}
\item A template is a "pattern" that the compiler uses to generate a family of classes or functions
\item Templates does not imply any run-time mechanisms
\end{itemize}

\begin{block}{Three kinds of templates}
\begin{itemize}
\item Function templates,
\item Class templates and,
\item Variable templates (since C++14)
\end{itemize}
\end{block}
\end{frame}

\section{Function Templates}
\begin{frame}[fragile]
\frametitle{Function Templates}

\begin{block}{Function Templates Syntax}
\begin{lstlisting}
template <class identifier> function_declaration;
template <typename identifier> function_declaration;
\end{lstlisting}
\end{block}

Both expressions have the same meaning and behaviour\\~\\

A type parameter need not be a class
\end{frame}

\subsection{Examples}
\begin{frame}[fragile]
\frametitle{make\_pair function template}

\begin{lstlisting}
template<typename T1, typename T2>
pair<T1,T2> make_pair(T1 a, T2 b)
{
    return {a,b};
}
auto x = make_pair(1,2); // x is a pair<int,int>
auto y = make_pair(string("New York"),7.7); // y is a pair<string,double>

\end{lstlisting}
\end{frame}


\begin{frame}
\frametitle{Template Instantiation}

\begin{block}{Template Instantiation}
The process of generating a class or a function from a template
\end{block}
\begin{itemize}
\item Function template arguments are deduced from the function arguments
\item If a template argument cannot be deduced from the function arguments, we must specify
it explicitly
    \begin{itemize}
    \item e.g: static\_cast , dynamic\_cast 
    \end{itemize}
\item Class template parameters are never deduced
\end{itemize}
\end{frame}


\begin{frame}[fragile]
\frametitle{max function template}

\begin{lstlisting}
template <typename T>
inline T max(T a, T b) {
    return a > b ? a : b;
}

max(3, 7)
max(3.0, 7.0)
max(3, 7.0) // error
max<double>(3, 7.0)
max(double(3), 7.0)
\end{lstlisting}

\end{frame}

\begin{frame}[fragile]
\frametitle{Template Specialization}
\begin{lstlisting}
template<typename T> 
inline std::string stringify(const T& x){
  std::ostringstream out;
  out << x;
  return out.str();
}

template<> 
inline std::string stringify<bool>(const bool& x) {
  std::ostringstream out;
  // return bools as "true" or "false" over "1" or "0" 
  out << std::boolalpha << x; 
  return out.str();
}

\end{lstlisting}

\end{frame}

\subsection{Function Template Overloading}
\begin{frame}
\frametitle{Function Template Overloading}

\begin{itemize}
\item Several function templates with the same name is possible
\item Combination of function templates and ordinary functions with the same name is possible
\end{itemize}

Overload resolution is used to find the right function or function template to invoke.

\end{frame}

\begin{frame}[fragile]
\frametitle{Overload Resolution Rules}

\begin{enumerate}
\item A non-templated overload is preferred to templates
\item Only non-template and primary template overloads participate in overload resolution
\item Specializations are not overloads and are not considered
\item Overload resolution selects the best-matching function template
    \begin{itemize}
        \item Now, specializations are examined to see if one is a better match
    \end{itemize}
\end{enumerate}
\end{frame}

\begin{frame}[fragile]
\frametitle{Overload resolution Examples}

\begin{block}{non-template vs overload templates}
\begin{lstlisting}
template <class T> void foo(T);
void foo(int);

foo(10); //calls void foo(int)
foo(10u); //calls void foo(T) with T = unsigned
\end{lstlisting}
\end{block}

\end{frame}

\begin{frame}[fragile]
\frametitle{Overload resolution Examples}

\begin{lstlisting}
template< class T > void f(T);   // #1: template overload
template< class T > void f(T*);  // #2: template overload
void   f(double);   // #3: nontemplate overload
template<> void f(int);  // #4: specialization of #1
 
f('a');        // calls #1
f(new int(1)); // calls #2
f(1.0);        // calls #3
f(1);          // calls #4
\end{lstlisting}
\end{frame}

\begin{frame}[fragile]
\frametitle{Argument Substitution Failure}

Substitution Failure Is Not An Error
\begin{lstlisting}

template<typename Iter> 
typename Iter::value_type mean(Iter first, Iter last); // #1

template<typename T> T mean(T*,T*); // #2

void f(vector<int>& v, int* p, int n)
{
    auto x = mean(v.begin(),v.end()); // OK: call #1
    auto y = mean(p,p+n); // OK: call #2
}

int*::value_type mean(int*,int*); // int* doesn't have such type
\end{lstlisting}
\end{frame}

\section{Class Templates}
\begin{frame}
\frametitle{Class Templates}

\begin{itemize}
\item Specification for generating classes based on parameters
\item Class generated from a class-template is a ordinary class
\end{itemize}

\begin{block}{Decrease of generated code}
Code for a member function of a class-template is only generated if that member is used
\end{block}

\end{frame}


\begin{frame}
\frametitle{Class Templates Declaration}

\begin{itemize}
\item Members declaration is same as non-template class
\item A template member need not be defined within the template class itself
\item Members of a template class are themselves templates
    \begin{itemize} 
    \item parameterized by the template class parameters
    \end{itemize}
\end{itemize}
\end{frame}

\subsection{Examples}

\begin{frame}[fragile]
\frametitle{Class Templates Examples}
\begin{lstlisting}
template <class T>
class mypair {
    T a, b;
  public:
    mypair (T first, T second) // defined in-class
      {a=first; b=second;}
    T getmax ();
};
template <class T>
T mypair<T>::getmax ()  { // defined outside-class
  T retval = a > b ? a : b;
  return retval;
}
mypair<int> myobject (100, 75); 

\end{lstlisting}

\end{frame}


\begin{frame}[fragile]
\frametitle{Non-type parameters for templates}
Templates can also have regular typed parameters.
\begin{lstlisting}
template <class T, int N>
class mysequence {
    T memblock [N];
  public:
    void setmember (int x, T value);
    T getmember (int x);
};

template <class T, int N>
void mysequence<T,N>::setmember (int x, T value) {
  memblock[x]=value;
}

\end{lstlisting}
\end{frame}

\begin{frame}[fragile]
\frametitle{Non-type parameters for templates}
\begin{lstlisting}
template <class T, int N>
T mysequence<T,N>::getmember (int x) {
  return memblock[x];
}

mysequence <int,5> myints;
mysequence <double,5> mydoubles;
myints.setmember (0,100);
mydoubles.setmember (3,3.1416);

\end{lstlisting}
\end{frame}

\begin{frame}[fragile]
\frametitle{Class Templates Overload}
It is not possible to overload a class template name.
\begin{lstlisting}
template<typename T>
class String { /* ... */ };

class String { /* ... */ }; // error : double definition

\end{lstlisting}
\end{frame}


\section{Type Checking}
\begin{frame}
\frametitle{Type Checking}

\begin{itemize}
\item Type checking is done on the code generated by template instantiation
\item Mismatch between what the programmer sees and what the compiler type checks can be a major problem
\item Errors that relate to the use of template parameters cannot be detected until the template is used.
\end{itemize}

\end{frame}

\begin{frame}[fragile]
\frametitle{Type Equivalence}

Aliases do not introduce new types.

\begin{lstlisting}
vector<unsigned char> s3;
using Uchar = unsigned char;
vector<Uchar> s4;
\end{lstlisting}

Both s3, s4 are instances of same-class

\end{frame}

\begin{frame}[fragile]
\frametitle{Type Equivalence}

\begin{itemize}
\item Types generated from a single template by different arguments are different types
    \begin{itemize} 
    \item Generated types from related arguments are not automatically related
    \end{itemize}

\item For example, assume that a Circle is a kind of Shape :
\begin{lstlisting}
Shape *p {new Circle(100)}; // Circle* converts to Shape*
vector<Shape> *q{new vector<Circle>{}}; 
// error : no vector<Circle>* to vector<Shape>* conversion
\end{lstlisting}
\end{itemize}

\end{frame}


\begin{frame}[fragile]
\frametitle{Template Aliases}

\begin{lstlisting}
template<typename T, typename Allocator = allocator<T>> vector;

using Cvec = vector<char>; // both arguments are bound
Cvec vc = {'a', 'b', 'c'}; // vc is a vector<char,allocator<char>
\end{lstlisting}

\begin{block} {In the standard library std::string is}
\begin{lstlisting}
using string = std::basic_string<char>
\end{lstlisting}
\end{block}

\end{frame}

\begin{frame}
\frametitle{Variable templates : since C++14}

\end{frame}


\begin{frame}[fragile]
\frametitle{Variadic templates (Since C++11)}
\begin{itemize}
\item Variadic templates takes variable number of arguments
\item Both function templates and class templates can be variadic
\end{itemize}
\begin{block} {Syntax}
\begin{lstlisting}
template<typename... Values> class tuple;  // takes zero or more arguments

template<typename First, typename... Rest> class tuple; // takes one or more arguments
\end{lstlisting}
\end{block}

\end{frame}

\section{Source Code Organization}
\begin{frame}[fragile]
\frametitle{Source Code Organization}

\begin{itemize}
\item Place the declaration and definition of the templates in the header file
\item Why can’t I separate the definition of my templates into a .cpp file ?
    \begin{itemize}
    \item Compiler has to see both the template definition (not just declaration) and the specific types for generating the code
    \end{itemize}
\end{itemize}

\end{frame}

\begin{frame}
\frametitle{Drawbacks to the use of templates}
\begin{itemize}
\item Many compilers lack clear instructions when they detect a template definition error.
\item It can be difficult to debug code that is developed using templates
\end{itemize}

\end{frame}

%------------------------------------------------

\section{References}
\begin{frame}
\frametitle{References}
\begin{thebibliography}{2} % Beamer does not support BibTeX so references must be inserted manually as below
\setbeamertemplate{bibliography item}[book]
\bibitem{Stroustrup} The C++ Programming Language [4th Edition] - Bjarne Stroustrup
\end{thebibliography}
\end{frame}

%------------------------------------------------

\begin{frame}
\Huge{\centerline{The End}}
\end{frame}

%----------------------------------------------------------------------------------------




%----------------------------------------------------------------------------------------

\end{document} 

