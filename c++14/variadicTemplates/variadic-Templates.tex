
%----------------------------------------------------------------------------------------
%	PACKAGES AND THEMES
%----------------------------------------------------------------------------------------

\documentclass{beamer}

\usetheme{Warsaw}

\usepackage{listings}
\usepackage[utf8]{inputenc}
\usepackage{url}
\setbeamertemplate{itemize item}[triangle]
%----------------------------------------------------------------------------------------
%	TITLE PAGE
%----------------------------------------------------------------------------------------

\title{Variadic Templates} % The short title appears at the bottom of every slide, the full title is only on the title page

\author{Venkatesh} % Your name
\institute{WDC}
\date{\today} % Date, can be changed to a custom date

\begin{document}

\lstset{breakatwhitespace=true,
language=C++,
columns=fullflexible,
keepspaces=true,
breaklines=true,
tabsize=3, 
showstringspaces=false,
extendedchars=true
inputencoding=utf8}


\begin{frame}
\titlepage % Print the title page as the first slide
\end{frame}

\begin{frame}{Outline}
  % You might wish to add the option [pausesections]
  \begin{enumerate}
   \item Introduction
   \item Fundamentals
        \begin{itemize}
        \item Parameter Packs
        \end{itemize}
   \item Examples
        \begin{itemize}
        \item Expansion loci
        \item Variadic Function template
        \item Tuples
        \end{itemize}
  \end{enumerate}
\end{frame}


%----------------------------------------------------------------------------------------
%	PRESENTATION SLIDES
%----------------------------------------------------------------------------------------

%----------------------------------------------------------------------------------------

\begin{frame}[fragile]
\frametitle{Introduction}

\begin{block}{Variadic templates}
mechanism for passing an arbitrary number of arguments of arbitrary types to a template
\end{block}

\end{frame}

\begin{frame}[fragile]
\frametitle{Variadic class template}

\begin{block}{e.g:}
\begin{lstlisting}
template<class ... Types> struct Tuple {};
Tuple<> t0;           // Types contains no arguments
Tuple<int> t1;        // Types contains one argument: int
Tuple<int, float> t2; // Types contains two arguments: int and float
Tuple<0> error;       // error: 0 is not a type
\end{lstlisting}
\end{block}

\end{frame}


\begin{frame}[fragile]
\frametitle{Variadic function template}


\begin{block}{e.g:}
\begin{lstlisting}
template<class ... Types> void f(Types ... args);
f();       // OK: args contains no arguments
f(1);      // OK: args contains one argument: int
f(2, 1.0); // OK: args contains two arguments: int and double
\end{lstlisting}
\end{block}

\end{frame}



\begin{frame}[fragile]
\frametitle{Fundamentals}

\begin{lstlisting}

template <typename... Ts>
class C
{
    // fill the body
};
 
template <typename... Ts>
void fun(const Ts&... vs)
{
    // fill the body
}

\end{lstlisting}

\end{frame}

\begin{frame}[fragile]
\frametitle{A New Kind: Parameter Packs}

Ts is not a type; vs is not a value!

\begin{lstlisting}
typedef Ts MyList; // error!
Ts var; // error!
auto copy = vs; // error!
\end{lstlisting}

\begin{itemize}
\item Ts is an alias for a list of types
\item vs is an alias for a list of values
\item Either list may be potentially empty
\item Both obey only specific actions
\end{itemize}

\end{frame}


\begin{frame}
\frametitle{Parameter pack}

\begin{block}{template parameter pack}
 is a template parameter that accepts zero or more template arguments
\end{block}

\begin{block}{function parameter pack}
 is a function parameter that accepts zero or more function arguments
\end{block}


\end{frame}


\begin{frame}[fragile]
\frametitle{Parameter pack Syntax}

\begin{block} {Template parameter pack}
\begin{lstlisting}
type ... Args(optional)

typename|class ... Args(optional)
\end{lstlisting}
\end{block}

\begin{block}{Function parameter pack}
\begin{lstlisting}
Args ... args(optional)
\end{lstlisting}
\end{block}

\end{frame}



\begin{frame}[fragile]
\frametitle{Constraint}

\textit{template parameter pack} must be the final parameter in the template parameter list 

\begin{block}{e.g:}
\begin{lstlisting}
template<typename... Ts, typename U> struct Invalid;
// Error: Ts.. not at the end
\end{lstlisting}
\end{block}
\end{frame}



\begin{frame}[fragile]
\frametitle{Using parameter packs}

\begin {itemize}
\item Apply sizeof... to it, this will return how many types in Ts
\begin{lstlisting}
size_t items = sizeof...(Ts); // or vs
\end{lstlisting}


\item {Parameter pack expansion}
\begin{block}{Syntax}
\begin{lstlisting}
pattern ...	
\end{lstlisting}
\end{block}
expands to comma-separated list of zero or more patterns.
Pattern must include at least one parameter pack
\end{itemize}

\end{frame}


\begin{frame}[fragile]
\frametitle{Pack expansion}

\begin{block}{e.g:}
\begin{lstlisting}
template<class ...Us> void f(Us... pargs) {}
template<class ...Ts> void g(Ts... args) {
    f(&args...); // "&args..." is a pack expansion
                 // "&args" is its pattern
}
g(1, 0.2, "a"); 
// Ts... args expand to int E1, double E2, const char* E3
// &args... expands to &E1, &E2, &E3
// Us... pargs expand to int* E1, double* E2, const char** E3
\end{lstlisting}
\end{block}

\end{frame}



\begin{frame}[fragile]
\frametitle{Expansion rules}

\begin{lstlisting}

Use            Expansion

Ts...          T1, .., Tn
Ts&&...        T1&&, .., Tn&&
x<Ts,Y>::z...  x<T1,Y>::z, .., x<Tn,Y>::z
x<Ts&,Us>...   x<T1&,U1>, .., x<Tn&,Un>
func(5,vs)...  func(5,v1), .., func(5,vn)

\end{lstlisting}
\begin{itemize}
\item When expanding two lists in lock-step, they should have the same size.
\end{itemize}
\end{frame}


\begin{frame}[fragile]
\frametitle{Multiple Expansions}

\begin{block}{Expansion proceeds inwards outwards}

\begin{lstlisting}
template <class ... Ts> void fun(Ts... vs) {

    gun(A<Ts...>::hun(vs)...);
    gun(A<Ts...>::hun(vs...));
    gun(A<Ts>::hun(vs)...); 

}
\end{lstlisting}
These are different expansions
\end{block}

\end{frame}


\begin{frame}[fragile]
\frametitle{Expansion loci }

1. Initializer lists

\begin{lstlisting}
any a[] = { vs... };
\end{lstlisting}

\begin{block}{example}
\begin{lstlisting}
template<typename... Ts> void func(Ts... args){
    const int size = sizeof...(args) + 2;
    int res[size] = {1,args...,2};
    int dummy[sizeof...(Ts)] = { (std::cout << args, 0)... };
}
\end{lstlisting}
\end{block}

\end{frame}


\begin{frame}[fragile]
\frametitle{Expansion loci }

2. Function argument lists (simply function-call operator)

\begin{block}{example}
\begin{lstlisting}
f(&args...); // expands to f(&E1, &E2, &E3)
f(n, ++args...); // expands to f(n, ++E1, ++E2, ++E3);
f(++args..., n); // expands to f(++E1, ++E2, ++E3, n);
f(h(args...) + args...); // expands to 
// f(h(E1,E2,E3) + E1, h(E1,E2,E3) + E2, h(E1,E2,E3) + E3)
\end{lstlisting}
\end{block}

\end{frame}

\begin{frame}[fragile]
\frametitle{Expansion loci }

3. Template argument lists

\begin{block}{example}
\begin{lstlisting}
template<class A, class B, class...C> void func(A arg1, B arg2, C...arg3)
{
    container<A,B,C...> t1;  
    // expands to container<A,B,E1,E2,E3> 
    container<C...,A,B> t2;  
    // expands to container<E1,E2,E3,A,B> 
    container<A,C...,B> t3;  
    // expands to container<A,E1,E2,E3,B> 
}
\end{lstlisting}
\end{block}

\end{frame}

\begin{frame}[fragile]
\frametitle{Expansion loci }

4. Base specifiers and member initializer lists

\begin{block}{example}
\begin{lstlisting}
template <typename... Ts>
struct C : Ts... {};

template <typename... Ts>
struct D : Box<Ts>... {  };

template<class... Mixins>
class X : public Mixins... {
 public:
    X(const Mixins&... mixins) : Mixins(mixins)... { }
};
\end{lstlisting}
\end{block}

\end{frame}


\begin{frame}[fragile]
\frametitle{Expansion loci }

5. Lambda captures

\begin{block}{example}
\begin{lstlisting}
template<class ...Args>
void f(Args... args) {
    auto lm = [&, args...] { return g(args...); };
    lm();
}
\end{lstlisting}
\end{block}

\end{frame}


\begin{frame}[fragile]
\frametitle{Expansion loci }

6. Function parameter list

\begin{block}{example}
\begin{lstlisting}
template<typename ...Ts> void f(Ts...) {}
f('a', 1);  // Ts... expands to void f(char, int)
f(0.1);     // Ts... expands to void f(double)
 
\end{lstlisting}
\end{block}

\end{frame}

\begin{frame}[fragile]
\frametitle{How to use variadic templates}

\begin{block}{usage approach}
Pattern Matching!
\end{block}

\end{frame}



\begin{frame}[fragile]
\frametitle{Examples : User defined Tuple }

\begin{lstlisting}
Tuple<double , int, char> x {1.1, 42, 'a'};
cout << x << "\n";
cout << get<1>(x) << "\n";
\end{lstlisting}

\end{frame}


\begin{frame}[fragile]
\frametitle{Advantages}

\begin{itemize}
\item Library utilities std::make\_unique, std::make\_shared
\end{itemize}

\begin{block}{std::make\_unique for single objects }
\begin{lstlisting}
   template<typename Tp, typename... Args>                             
   make_unique(Args&&... args) { 
        return unique_ptr<Tp>(new Tp(std::forward<Args>(args)...));
   }
\end{lstlisting}
\end{block}

\end{frame}




%------------------------------------------------
\begin{frame}
\frametitle{References}
\begin{thebibliography}{2} % Beamer does not support BibTeX so references must be inserted manually as below
\setbeamertemplate{bibliography item}[book]
\bibitem{Stroustrup} The C++ Programming Language [4th Edition] - Bjarne Stroustrup
\end{thebibliography}
\end{frame}

%------------------------------------------------

\begin{frame}
\Huge{\centerline{Thank You}}
\end{frame}

%----------------------------------------------------------------------------------------




%----------------------------------------------------------------------------------------

\end{document} 

